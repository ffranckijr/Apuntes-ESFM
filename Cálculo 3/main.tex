\documentclass{book}
\usepackage[spanish]{babel}
\usepackage[utf8]{inputenc}
\usepackage{mathrsfs}
\begin{document}
\begin{titlepage}
\centering
{\bfseries\LARGE INSTITUTO POLITÉCNICO NACIONAL \par}
\vspace{1cm}
{\scshape\Large Escuela Superior de Física y Matemáticas \par}
\vspace{3cm}
{\scshape\Huge Apuntes de Cálculo 3 \par}
\vspace{3cm}
{\itshape\Large Imparte la Dra. Laura Roció Gózalez  \par}
\vfill
{\Large Autor: \par}
{\Large Francisco Alexis Franco Camacho \par}
\vfill
{\Large Febrero 2023 \par}
\end{titlepage}
\tableofcontents
\chapter{Introducción}
\section{Objetivo}
Cálculo diferencial en varias variables de manera teórica y con aplicaciones.
\section{Temario}
\begin{itemize}
\item $R^n$ como espacio euclidiano
\item Norma, distancia y desigualdad del triangulo.
\item Conjuntos abiertos, cerrados.
\item Conexidad.
\item Sucesiones en $R^n$.
\item Convergencia, compacidad.
\item Teorema de Bolzano-weirstrass.*
\item Teorema de Heine-Borel.*
*Propiedades de compacidad.
\item Limite de transformaciones.
\item Continuidad de transformación.
\item Continuidad de inversa de transformación.
\item La diferencial de una transformación.
\item Transformaciones diferenciales.
\item Regla de la cadena.
\item Derivada direccional.
\item Funciones clase $C^n$.
\item Teorema de función inversa y Teorema de función implícita.
\item Diferenciales de orden superior.
\item Teorema de Taylor. Aplicaciones a máximos y mínimos.
\end{itemize}
\section{Bibliografía}
\begin{itemize}
    \item Elementary Classical Analysis-Marsden and Hoffman.
    \item Mathematical Analysis, Apostol.
    \item Analysis on manifolds, Munkres.
    \item Mathematical Analysis, Rudin.*
    \item Calculus on manifolds, Spivak.*
    *Densos,
\end{itemize}
\section{Evaluación}
\begin{itemize}
    \item Primer Parcial $25\%$.
      \item Segundo Parcial $25\%$.
        \item Tercer Parcial $25\%$.
        \item Quizes $25\%$.
\end{itemize}
\subsection{Quizes}
\begin{itemize}
    \item Son de opción múltiple.
    \item Son sorpresa.
    \item Se elimina el Quiz que tenga la calificación mas baja.
    \item Se saca promedio al final del semestre.

\end{itemize}

\chapter{$R^n$ como espacio euclidiano.}
\section{El espacio $R^n$.}
Se define el n-espacio euclidiano de n-tuplas en R como:
$$R^n=\{(x_{1},x_{2},...,x_{n}) \hspace{0.2cm} | \hspace{0.2cm} x_{i} \in R, 1\leq i\leq n\}$$
Es decir:
$$R^n=R*R*R*...*R$$
Sea:
$$\overrightarrow{x}\in R^n$$ 
Entonces:
$  \overrightarrow{x}$ es un punto en $R^n$ o un vector en el R-espacio vectorial.

\subsection{Definición de la suma y multiplicación por escalar.}
Sea:
$$\overrightarrow{x}=(x_{1},x_{2},...,x_{n})\in R^n$$
$$\overrightarrow{y}=(y_{1},y_{2},...,y_{n})\in R^n$$
$$\alpha \in R$$
Se define la suma:
$$\overrightarrow{x}+\overrightarrow{y}:=(x_{1}+y_{1},x_{2}+y_{2},...,x_{n}+y_{n})$$
Se define la multiplicación por escalar:
$$\alpha\overrightarrow{x}:=(\alpha x_{1},\alpha x_{2},...,\alpha x_{n})$$
\subsection{Tarea}
Demostrar que $(R^n,+,*)$ es un R-espacio vectorial de dimensión n.
\subsubsection{Demostración de que $R^n$ es un espacio vectorial}
Sean: $\overrightarrow{x},\overrightarrow{y},\overrightarrow{z}\in R^n$
\subsubsection{Demostración de que su dimensión es n}
\section{Hipótesis}
\chapter{Resultados}
\section{Simulación de resultados}
\subsection{Suposiciones}
\subsection{Modelos}
\section{Resultados preliminares}
\section{Resultados postprocesados}
\subsection{Valores atípicos}
\subsection{Correlaciones}
\chapter{Conclusiones}
\end{document}