\documentclass{book}
\usepackage[spanish]{babel}
\usepackage[utf8]{inputenc}
\usepackage{mathrsfs}
\usepackage{amsmath}
\begin{document}
\begin{titlepage}
\centering
{\bfseries\LARGE INSTITUTO POLITÉCNICO NACIONAL \par}
\vspace{1cm}
{\scshape\Large Escuela Superior de Física y Matemáticas \par}
\vspace{3cm}
{\scshape\Huge Apuntes de Álgebra 3 \par}
\vspace{3cm}
{\itshape\Large Imparte el Dr. Joel Cruz Ramírez  \par}
\vfill
{\Large Autor: \par}
{\Large Francisco Alexis Franco Camacho \par}
\vfill
{\Large Febrero 2023 \par}
\end{titlepage}
\tableofcontents
\chapter{Introducción}
\section{Objetivo}
No dieron objetivo
\section{Temario}
\begin{itemize}
\item Transformaciones ortogonales.
\item Funciones lineales y operadores adjuntos.
\item Productos interno y operadores positivos.

\end{itemize}
\section{Bibliografía}
\begin{itemize}
    \item Elements of linearl Algebra and matrix theory Moore; Mc. Graw Hill.
    \item Elements of linearl Algebra L.J Paige and J.D Swift; Blaisdell.
    \item Lectures on Linear Algebra Gelfand; Interscience.
    \item Foundations of linearl Algebra A.I Malcer; Freeman.*
    *Densos,
\end{itemize}
\section{Evaluación}
\begin{itemize}
    \item Primer Parcial $33\%$.
    \item Segundo Parcial $33\%$.
    \item Tercer Parcial $33\%$.
        
\end{itemize}


\chapter{Transformaciones Ortogonales}
\section{Operadores Ortogonales}
Definición:
Un operador lineal $T$ en un espacio euclidiano $V$ es Ortogonal si $(T\alpha,T\alpha)=(\alpha,\alpha)\hspace{0.2cm}$ para cualquier $\alpha \in V$.
\\Esto es un Operador lineal Ortogonal si se conserva la distancia de todo vector en el espacio.


\subsection{Ejemplo/Ejercicio.}
Sea: $T:R^2\longrightarrow R^2$ Ortogonal, Transformación lineal.
\\$R^2$ Con el producto estándar para cada $X\in R^2$
\\Sea $(x,y)\in R^2$ y $T(x,y)=(ax + by,cx + dy )$
\\Operando Producto Interno:
\\$(T(x,y),T(x,y))=((ax+by,cx+dy),(ax+by,cx+dy))$
$\\\hspace*{2.5cm}=(ax+by)(ax+by)+(cx+dy)(cx+dy)$
$\\\hspace*{2.5cm}=a^2x^2+2axby+b^2y^2+c^2x^2+2cdxy+d^2y^2$
$\\\hspace*{2.5cm}=(a^2+c^2)x^2+2(ab+cd)xy+(b^2+d^2)y^2$
$\\\hspace*{2.5cm}=x^2+y^2\Longleftrightarrow a^2+c^2=1$
$\\\hspace*{2.5cm}\Longleftrightarrow 2(ab+cd)=0$
$\\\hspace*{2.5cm}\Longleftrightarrow b^2+d^2=1$
\\Tomamos la base canónica ${(1,0),(0,1)}$
\\Entonces
$\\T(1,0)=(a*1+b*0,c*1+d*0)=(a,c)$
$\\T(0,1)=(a*0+b*1,c*0+d*1)=(b,d)$
$\\((1,0),(0,1))=0$ es decir que es ortogonal
\\Por lo tanto:
\\$((1,0),(0,1))=(T(1,0),T(0,1))=((a,c),(b,d))=ab+cd=0$
\\El análisis siguiente lo haremos por casos:
\begin{enumerate}
    \item Cuando $ab+cd=0$ y al menos a o b son 0 lo que implica que c o d son 0. 
    $\\-cd=ab$
    \begin{enumerate}
        \item Caso: b=0 y d=0:
        $\longrightarrow b^2+d^2=1$ sustituyendo $0^2+d^2=1$ lo que es una incongruencia, por lo tanto descartamos este caso.
        \item Caso: b=0 y c=0:
        $\\\longrightarrow a^2+c^2=1$ sustituyendo $\longrightarrow a^2+0^2=1 \longrightarrow a^2=1 \longrightarrow a=\pm1$
        $\\\longrightarrow b^2+d^2=1$ sustituyendo $\longrightarrow 0^2+d^2=1 \longrightarrow d^2=1 \longrightarrow d=\pm1$
        \\Entonces Usando esto en la transformación lineal:
        \begin{enumerate}
            \item a=1, b=0, c=0, d=1:
            $\\T(x,y)=(x+0y,0x+y)=(x,y)$ entonces es un operador ortogonal.
            \item a=-1, b=0, c=0, d=1:
            $\\T(x,y)=(-x,y)$ entonces no es un operador ortogonal.
            \item a=1, b=0, c=0, d=-1:
            $\\T(x,y)=(x,-y)$ entonces no es un operador ortogonal.
            \item a=-1, b=0, c=0, d=-1:
            $\\T(x,y)=(-x,-y)$ entonces no es un operador ortogonal.
            
        \end{enumerate}
        
    \end{enumerate}
    \item Cuando $ab+cd=0$ pero a, b, c ,d distintos de 0
    $\\ \longrightarrow-\frac{a}{c}=\frac{d}{b}=\lambda$
    \\$-\frac{a}{c}=\lambda\longrightarrow a=-\lambda c$
    \\$d=\lambda b$
    \\ Sustituimos:
    $\\a^2+c^2=1\longrightarrow \lambda^2c^2+c^2=1\longrightarrow c^2(\lambda^2+1)=1$
    $\\b^2+d^2=1\longrightarrow b^2+\lambda^2b^2=1\longrightarrow b^2(1+\lambda^2)=1$
    \\Dividimos las expresiones divididas tal que así:
    \\$\frac{c^2}{b^2}=1\longrightarrow c^2=b^2\longrightarrow b=\pm c$
    \begin{enumerate}
        \item b=c:
        \\$ab+cd=0$ sustituyendo: $a(c)+cd=0 \longrightarrow a=-d $
        \\$T(x,y)=(ax+by,bx-ay)$
        \item b=-c:
        \\$ab+cd=0$ sustituyendo: $a(-c)+cd=0 \longrightarrow a=d $
        \\$T(x,y)=(ax-by,-bx+ay)$
    \end{enumerate}
    \item Podemos obtener también la representación matricial:
    \\
    \begin{equation}
      [T]=
        \begin{pmatrix}
            a & b\\
            b & -a
        \end{pmatrix}
        \begin{pmatrix}
            x\\
            y
        \end{pmatrix}
        =
        \begin{pmatrix}
            a & -b\\
            -b & a
        \end{pmatrix}
    \end{equation}
\end{enumerate}
\newpage
\subsection{Tarea}
Demostrar que $(R^n,+,*)$ es un R-espacio vectorial de dimensión n.
\subsubsection{Demostración de que $R^n$ es un espacio vectorial}
Sean: $\overrightarrow{x},\overrightarrow{y},\overrightarrow{z}\in R^n$
\subsubsection{Demostración de que su dimensión es n}
\section{Hipótesis}
\chapter{Resultados}
\section{Simulación de resultados}
\subsection{Suposiciones}
\subsection{Modelos}
\section{Resultados preliminares}
\section{Resultados postprocesados}
\subsection{Valores atípicos}
\subsection{Correlaciones}
\chapter{Conclusiones}
\end{document}